\documentclass[a4paper,10pt]{article}
\usepackage[utf8]{inputenc}

%opening
\title{Manifesto}
\author{Pierre Fernagu--Berthier}

\begin{document}

\maketitle

\paragraph{Introduction}

As modern day IT spreads around our world, major innovation has been achieved. But not all of them are actually great, and a part of them a regression for mankind. As such, in this manifesto, I will expose my vision of what should modern IT be.

\section{The place of everyday people}

\paragraph{Computers}

As an IT hobbyist, I don't believe that computers as they exist today should be a necessity for normal folks. When those devices were created back in the days, they were only accessible to a few people which knew well the tools they were using. But now, most computers users don't even know the major principles of computing. Some of them see it as a unknown tool forced unto them, others only grab a little bit of it. It then creates issues like viruses that spread like wildfire. Such issues mostly exist because the majority of modern computers' users don't understand what a virus is, what it does, how it spreads?

I believe that specific tools, using IT technologies, should be available to the wider public such as productivity machines that only do productivity tasks. You can imagine as many of those specific machines as you want, but they all should be either very accessible or very complex. They should also be easily interoperable between them, file format should be able to be open on the simpler machine as well as on the harder machine. A part of those devices already exist, the 9-keys phone is a great example, it gets out of your way, not making you less productive and concentrated.

\paragraph{Internet}

From my point of view, normal people should not have easy access the wider Internet. Internet is a fantastic tool to find information, learn, share and create but it is also a superb tool to scam, pirate and attack a naive and unacknowledged person. I think that they should have access to Internet but in a software that constrains the ability of the person to be scamed. I will like to give the example that maybe Wikipedia, should be accessible through a secure interface that should take the form of a software; in that software, people should be able to read, modify and check the article contains in the different pages of the website such as Wikipedia, Wikidictionary or Wikidata.

\section{Internet}

\paragraph{A convoluted Internet}

Internet is an amazing tool, as said higher, but it has become convoluted with over complex websites that do way too many tasks. You should not be able to have something like Snapchat, that serves as a messaging app, a photo and video sharing platform, a camera software, a localizing software and much more. Internet websites should be respecting the Unix philosophy, Twitter should only be a short blogging platform, Instagram should only be a photo sharing platform, Google should only do web-searches. Those websites should also be designed to be able to run on any computer, even older ones. They should be designed around universally adopted standards and should be plain simple when possible.

\paragraph{Decentralize}

Internet should also be centralized around major datacenters owned by big corporations like Amazon, Google, Microsoft, etc. For example, when the OVH data centre in Strasbourg caught fire in 2020, most important French websites went off. Decentralizing the Internet through protocols like Torrent or systems like the Fediverse should help prevent such events being as impactful. But it could also help prevent some corporations from dominating the Internet and also being in a near monopoly like Google does on web-searches.

\paragraph{The software}

Internet is mainly built around open-standards, which is a good thing as no company really has the monopoly on Internet. But it doesn't take away the fact that some monopolies do exist like Chromium-based web browser which allows Google to enforce major choices that only contribute to their own interest. One great example of that is how Google can just make sure that their services work way better on their own technology but also that some of the features are only working on their technologies. This should not be a thing on the Internet as it should be a fully open and free place.

\pagebreak

\section{Communication}

\paragraph{Messaging platforms}

As people discover the marvellous joy of being able to communicate easily and instantly with people on the other side of the globe, some of the basic privacy rights are infringed. When it was created WhatsApp was an amazing thing, you could communicate as easily as with other major platforms but it was also very secure and private. Unfortunately, WhatsApp is maintained by a private company which in 20** was bought by Facebook, one of the biggest offenders of privacy. The more time passes, the more WhatsApp becomes less and less private. I have concluded that major communication platforms should be maintained by non-profit organizations like it is the case with Signal. This would allow perfectly private communication. As said before, I believe that those communication platforms should respect the closest the Unix philosophy. A messaging app, should only do messaging and maybe calls but that's all.

\paragraph{Social media platforms}

Social media have made my generation dumber. They scroll endlessly, all day long, on infinite feed; and when they are bored looking at photos on Instagram, they go on TikTok to scroll through the infinite void of short videos. People socialize less, being cool is being quite active on the social media and if you don't check them multiple times a day, you can be framed to be uninformed. I don't think any of this has anything good for us. They have a break or free-time, they take out their "smart"phones to scroll in the void and remember nothing of what they had just saw.

First, having access to those social media on a pocket-able device make them even more addict. This is the reason why I advocate for simpler phones that don't have access to those social media. Secondly, I don't think that minors (or sixteen years old) should have any access to those. They don't bring anything constructive to them, make those children code, read Wikipedia, or even play video games when using computers. Finally, the less social media you expose yourself to, the better you feel. So making full weeks or even months without social media should help our brains take a break. Unfortunately, in our modern society completely dropping them is an impossible task.

\pagebreak

\section{Operating system}

\paragraph{Choice of OS}

Every operating system should be considered for its benefits and drawbacks. Users should, if necessary, should use the operating system that suits their work-flow the best. Nonetheless, if possible a free and open-source operating system should be preferred, even more so if it is POSIX-compliant.

\paragraph{Technically superior OS}

What defines a technically superior operating system is first of all if it respects the Unix philosophy. Secondly, it should be POSIX-compliant. Thirdly, it should be free and open-source. The type of kernel doesn't, from my point of view, affect the quality of the operating system; they all have their own advantages and disadvantages. The choice of the file-system should not affect the quality of the operating system unless it is quite old. A simple by default, yet powerful when needed is one great philosophy for the overall design of the operating system. Designing an operating system around specific software, if done right, is a good thing as it should make the operating system more efficient and optimized.

\section{Hardware}

\paragraph{Right to repair}

Right to repair is one of the most important rights in IT as it is associated with the right of ownership. You cannot really own thy device if you aren't able to repair it when you break it or it gets used like the battery that will deteriorate with time. Companies refusing or ignoring the right to repair, like Apple or Samsung, should not be encouraged nor accept little marketing steps that don't help right to repair go forward.

\pagebreak

\section{Software}

\paragraph{Free and open-source}

Most software should benefit from being open-source. It allows the community around it to propose bug and security fixes, new features through pull requests and it creates the ability to check for unwanted chances. Red Hat is a great example of monetizing free and open-source software while supporting the community around it and the projects used. Such entreprise support helps the free and open-source world to develop much faster and to be more secure. Nonetheless, in some cases, free and open-source software is the solution, and as long as the developer or the company maintaining the project stays very transparent about what is going on, it should be fine.

\paragraph{Usability}

As said previously, a software should be "simple by default, yet powerful when needed". Some software that are designed for professional use can understandably be designed around being powerful and in those cases, it isn't much of an issue. When designing software for the general public, accessibility should be one of the most important parts of the design of the operating system. Every software project should include a way for the community to help translating the software, it is one of the advantages of open-source, it is way easier for the community to set-up the infrastructure for such endeavour. The software should be designed with the Unix philosophy in mind whenever possible.

\paragraph{Graphic design}

Graphic design should not be one of the most prioritized parts of designing software, as long as the software is legible, usable, and intuitive. It is only a nice to have once the software is usable and stable enough. Software consistency is only of the utmost importance if the software is designed around a suite of software like the KDE project. Not wanting thy software to be themed by the user is fine, and comprehensible, as long as a light/dark theme and colour accents options are given.

\end{document}
